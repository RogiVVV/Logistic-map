\documentclass{article}
\usepackage{graphicx} 
\usepackage[utf8]{inputenc}
\usepackage[T2A]{fontenc}
\usepackage[russian]{babel}
\usepackage{graphicx} 


\begin{document}

\maketitle

\section{Easy}
\subsection{}
Рассмотрим рекуррентное соотношение
\[
x_{n+1} = r x_n (1 - x_n), \qquad r \in (0,1], \qquad 0 < x_0 < 1.
\]
Докажем по индукции по $n$, что
\[
\forall n \in \mathbb{N}\quad 0 < x_n < 1.
\]

\textbf{База:}
при $n=1$: т.к. $0 < x_0 < 1$, $0 < r \le 1$, то $rx_n(1-x_n)$ - произведение трёх положительных чисел $<1$. Значит оно $\in (0,1]$

\textbf{Индукционный переход:}
пусть для некоторого $n \in \mathbb{N}$ выполнено
\[
0 < x_n < 1.
\]
Покажем, что тогда $0 < x_{n+1} < 1$.

Из неравенства $0 < x_n < 1$ следует $0 < 1 - x_n < 1$.
Тогда произведение трёх положительных чисел $<1$. Значит оно $\in (0,1]$:
\[
0 < x_{n+1} = r x_n (1 - x_n) < 1.
\]

Таким образом, из выполнения $0 < x_n < 1$ следует выполнение $0 < x_{n+1} < 1$. По принципу математической индукции
\[
\forall n \in \mathbb{N} \quad 0 < x_n < 1.
\]

\section{Normal}
\subsection{}
Фиксированная точка $x^*$ удовлетворяет уравнению
\[
x^* = f_r(x^*) = r x^*(1 - x^*).
\]
Перенесём всё в одну сторону:
\[
r x^*(1 - x^*) - x^* = 0
\;\Rightarrow\;
x^* \bigl( r(1 - x^*) - 1 \bigr) = 0.
\]
Отсюда два варианта:
\[
x^* = 0
\]
(это фиксированная точка при любом $r$)
или
\[
r(1 - x^*) - 1 = 0
\;\Rightarrow\;
1 - x^* = \frac{1}{r}
\;\Rightarrow\;
x^* = 1 - \frac{1}{r}.
\]
То есть множество неподвижных точек:
\[
\left\{\,0,\; 1 - \frac{1}{r}\,\right\}.
\]
Так для любого $r \in (0,1]$ существуют ровно две неподвижные точки

\bigskip
\subsection{}
Рассмотрим отношение соседних членов логистического отображения:
\[
\frac{x_n}{x_{n+1}}
= \frac{x_n}{r x_n (1 - x_n)}
= \frac{1}{r(1 - x_n)}.
\]
Так как \(r \in (0,1]\) и \((1 - x_n) \in (0,1)\), то
\[
\frac{1}{r(1 - x_n)} > 1,
\]
следовательно,
\[
x_n > x_{n+1}\; \forall n,
\]
то есть последовательность \(\{x_n\}\) монотонно убывает, чтд.

\bigskip
\subsection{}

Пусть \(r \in (2,3)\), \(x_{2n} > x^*\), \(x_{2n+1} < x^*\). Тогда обе подпоследовательности
\(\{x_{2n}\}\) и \(\{x_{2n+1}\}\) сходятся к \(x^*\), причём чем больше \(r\),
тем более плавным становится их сближение с \(x^*\).

\textbf{Док-во:}

Рассмотрим вторую итерацию функции:
$f^2(x) = f(f(x)) = r^2 x (x - 1) (r x (x - 1) + 1)$. Производная $(f^2)'(x^*) = (f'(f(x^*))) f'(x^*) = (f'(x^*))^2 = (2 - r)^2$
Т.к. $ r \in (2,3)$, то $(2-r) \in (-1, 0)$, а $(2-r)^2 \in (0, 1)$
следовательно, $f^2$ сжимает окрестность $x^*$. То есть чётная подпоследовательность $\{x_{2n}\}$ монотонно убывает к $x^*$ (поскольку $x_{2n} > x^*$ и $f^2(x_{2n}) < x_{2n}$ для $x_{2n} > x^*$), а нечетная $\{x_{2n+1}\}$ монотонно возрастает к $x^*$ (аналогично).

\begin{flushright}
чтд.
\end{flushright}

\bigskip
\subsection{}

Рассматривается отображение
\[
x_{n+1} = g(x_n) = r x_n (1 - x_n)(2 + x_n),
\qquad r \in \left[0;\, \frac{27}{2(7\sqrt{7}-10)}\right].
\]

\textbf{1. Неподвижные точки}

Неподвижная точка \(x^*\) удовлетворяет уравнению
\[
x^* = r x^* (1 - x^*)(2 + x^*).
\]
Перенесём всё в одну сторону:
\[
r x^* (1 - x^*)(2 + x^*) - x^* = 0
\;\Rightarrow\;
x^*\bigl(r(1 - x^*)(2 + x^*) - 1\bigr) = 0.
\]
Отсюда получаем
\[
x^*_1 = 0,
\]
а остальные неподвижные точки являются корнями квадратного уравнения
\[
r(1 - x^*)(2 + x^*) - 1 = 0
\;\Leftrightarrow\;
r(2 + x^* - 2(x^*)^2) = 1.
\]
То есть всего может быть 1-3 неподвижных точек.
\medskip

\textbf{2. Диапазон параметра \(r\), при котором \(x_n\) монотонно сходится к нулю}

Для монотонного убывания к 0 нужно, чтобы $x_{n+1} < x_n \;\forall n$

Это соблюдается при всех r \in (0, 1)

\section{Hard}

\subsection{}
При $r=3$ получается цикл длины 2. При увеличении r количество точек в цикле увеличивается в 2 раза (на графике мы можем явно это видеть)

Таким образом m может принимать только значения степеней двойки, а при $r=r_{\infty}$ достигается своего рода хаотичность, нельзя выделить конкретный цикл

\end{document}
