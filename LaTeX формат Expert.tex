\documentclass{article}
\usepackage{graphicx}
\usepackage[utf8]{inputenc}
\usepackage[T2A]{fontenc}
\usepackage[russian]{babel}
% Required for inserting images

\begin{document}

\section{Expert}

\subsection{Следует ли асимптотическая устойчивость $x^∗$ из условия:}
$\exists \delta_0 > 0:\ |x_0 - x^*| < \delta_0 \;\Rightarrow\; x_n \rightarrow x^*$ при $n\to\infty$?
\newline
\newline
Допустим точка не устойчивая. Это значит, что 
\[
\exists \varepsilon > 0:\ \forall \delta > 0\ \exists n:
|x_0 - x^*| < \delta \;\Rightarrow\; |x_n - x^*| \ge \varepsilon.
\]
То есть $x_n \not\rightarrow x^*$ при $n\to\infty$ - противоречие. Значит данного условия не достаточно.

\subsection{Докажите или опровергните утверждение:}
В логистическом отображении при $r \in (0; 1)$ неподвижная точка $x^* = 0$ является устойчивой. Является ли она асимптотически устойчивой?

\bigskip

Рассмотрим отображение:
\[
x_{n+1} = r x_n (1 - x_n), \quad r \in (0,1), \quad x_0 \in [0,1].
\]
Последовательность $\{x_n\}$ неотрицательна и убывает, значит он имеет предел

\[
\lim_{n\to\infty} x_n = L \ge 0.
\]

Переходя к пределу в рекуррентном соотношении, получаем
\[
L = r L (1 - L).
\]
Переносим всё в одну сторону:
\[
r L (1 - L) - L = 0 \quad\Longleftrightarrow\quad
L \bigl(r(1 - L) - 1\bigr) = 0.
\]
Следовательно,

$L = 0$ или $ r(1 - L) - 1 = 0$
Из $r(1 - L) - 1 = 0$ имеем

\[
r(1 - L) = 1 \quad\Longrightarrow\quad 1 - L = \frac{1}{r} \ge 1
\quad\Longrightarrow\quad L \le 0.
\]
Но $L \ge 0$, поэтому возможно только $L = 0$.

Таким образом,
\[
\lim_{n\to\infty} x_n = 0
\]
для любого $x_0 \in [0,1]$ и любого $r \in (0; 1)$
\subsection{Докажите, что точка $x^* = 0$ при $r \in (2; 3)$ является неустойчивой}

Проверим условие неустойчивости:
\[
\exists \varepsilon > 0:\ \forall \delta > 0\ \exists n:
|x_0 - x^*| < \delta \;\Rightarrow\; |x_n - x^*| \ge \varepsilon.
\]
И правда, для достаточно большого n найдётся бесконечно малая $\varepsilon$, удовлетворяющая условию, так как при $r \in (2; 3)$ $x_n$ будут расти


\subsection{С помощью внешних источников исследуйте: как связано наличие цикла с периодом 3 с хаотичностью системы?}

Наличие цикла периода 3 — это «маркер» хаоса: если непрерывное отображение на отрезке имеет цикл порядка 3, то его динамика уже содержит в себе хаотическое поведение в строгом математическом смысле.

Для непрерывного отображения
f:[a,b]→[a,b] (сюда попадает и логистическое отображение) верна теорема Ли–Йорка: если у f есть периодическая орбита периода 3, то:
\newline
1) Существуют периодические орбиты всех целых периодов
n = 1, 2, 3,... 
\newline
2) существует неконтинуум точек, траектории которых не являются периодическими и не стремятся к периодическим орбитам (такие орбиты называют хаотическими)

\subsection{}
\end{document}
